\chapter{Template Features}
This chapter gives examples on what you can do with this template. It's just a brief overview. Please consult the common sources on how to write scientific documents and documents with \LaTeX.

\section{Structure}
This template provides three structural levels that appear in the table of contents: \texttt{\textbackslash chapter}, \texttt{\textbackslash section}, and \texttt{\textbackslash subsection}. Chapters will always start on a new page. Additionally, you can use \texttt{\textbackslash subsubsection} and \texttt{\textbackslash paragraph} as non-hierarchical means to structure your thesis.


\subsection{Lists}
You can use the default \LaTeX \- functions for writing lists, viz., \texttt{\textbackslash enumerate} for numbered lists and \texttt{\textbackslash itemize} for bullet point lists. Again, the \texttt{\textbackslash subsubsection} and \texttt{\textbackslash paragraph} can be used as structural elements, e.g., when listing definitions of terms.

Lists with points and dashes
\begin{itemize}
\item Point 1
\item Point 2
    \begin{itemize}
    \item Point 1
    \item Point 2
    \end{itemize}
\end{itemize}

or numbered
\begin{enumerate}
\item Point 1
\item Point 2
\end{enumerate}

\subsection{Footnotes}
Footnotes are continuously numbered throughout the document. Use the \texttt{\textbackslash footnote\{text\}} command.  They appear on the page their reference is on \footnote{This is an exemplary footnote.}. Footnotes have to be placed without whitespace behind the word and within the sentence boundaries, i.e., before the period.

\subsection{ToDo-Notes}
You can use ToDo notes using the \texttt{\textbackslash todo\{text\}}  command. Please make sure to remove any ToDo notes before handing in your thesis! \todo[inline]{ToDo: Remove me before publishing}

\subsection{Abbreviations}
You can use abbreviations like these: \gls{CAM}. Note that on the first occurrence, both the acronym and the full wording are displayed. After that, each occurrence uses the abbreviation: \gls{CAM}. A longer example: \gls{Grad-CAM}.

\section{Formatting Text}
\LaTeX \- provides \texttt{\textbackslash textit\{text\}} for \textit{italics}, \texttt{\textbackslash textbf\{text\}} for \textbf{bold face}, \texttt{\textbackslash texttt\{text\}} for \texttt{typewriter}, \texttt{\textbackslash textsc\{text\}} for \textsc{small caps}, \texttt{\textbackslash underline\{text\}} for \underline{underline}. Additionally, the template provides  \texttt{\textbackslash texthl\{text\}} for \texthl{highlighted text}. Please remove any highlighted text before handing in your thesis!

\section{Tables}

\begin{table}[caption={Table Example 1}, label=tab:table1]
    \centering
        \begin{tabular}{|l|cc|}
            \hline
            \hline
                a & b & c \\
            \hline
            \hline
                a & y  & z \\
            \hline
                1 & 2 & 3 \\
            \hline
            \hline
        \end{tabular}
\end{table}

\begin{table}[caption={Table Example 2}, label=tab:table2]
	\centering
		\begin{tabular}{c c c}
			\toprule
			{\bf A} & {\bf B} & {\bf C} \\ \midrule
			    x & y & z \\
                1 & 2 & 3 \\
                4 & 5 & 6 \\
			\midrule
		\end{tabular}
\end{table}

\section{Figures}

The \texttt{figure} environment is wrapped around images. These images should either be included as PDF-file via \texttt{\textbackslash includegraphics}, or created via \textit{TikZ/PGF}. For included images, make sure to use high-resolution images, preferably vector images.

Figures float, i.e., they do not necessarily appear at exact the same position you have defined them. Make sure to set a  \textit{caption} and an optional \textit{label} as figure parameters. 

\begin{figure}[label={fig:fig1}, caption={Anki Lab Logo}]
    \includegraphics[width=.6\textwidth]{figures/anki_logo.png} 
\end{figure}

Another possibility is to use non-floating figures. Figure \ref{fig:fig2} is an example of this. For more information, see \cite{forcefigureplacement}.

\begin{center}
    \includegraphics[width=.6\textwidth]{figures/anki_logo.png} 
    \captionof{figure}{Anki Lab Logo}
    \label{fig:fig2}
\end{center}


\subsection{Subfigures}
Sometimes it might be handy to contrast figures, i.e., by placing them next to each other. The template uses the \textit{subcaption} package to provide subfigures. The following example contains two figures, where each subfigure has its own \texttt{\textbackslash label} and \texttt{\textbackslash caption}. Additionally, the whole figure has its own \textit{caption} and \textit{label}. That means, you can reference subfigures  Figure \ref{fig:subfig1} and Figure  \ref{fig:subfig}.

Subfigures are not limited to images, but may also include listings or tables. Figure \ref{fig:subfig} shows a sample database query expressed in SQL (Figure \ref{fig:subfig1}) and as query plan in relational algebra  (Figure \ref{fig:subfig2}).
 
\begin{figure}[caption={Exemplary use of subfigures}, label={fig:subfig}]
	
	\begin{subfigure}[b]{.45\textwidth}
		
		\begin{lstlisting}[nolol, language=SQL]
		SELECT b, d FROM 
			EXAMPLE.RELATION1 r,
			EXAMPLE.RELATION2 s,
		WHERE 
			r.a = 'c'
		AND 
			s.e = 2
		AND 
			r.c = s.c; 
		\end{lstlisting}
		\caption{SQL select statement}\label{fig:subfig1}
	\end{subfigure}
	\begin{subfigure}[b]{.53\textwidth}
		\centering	
		\begin{tikzpicture}[node distance = 2cm, auto,
		database/.style={
			cylinder,
			cylinder uses custom fill,
			cylinder body fill=gray!30,
			cylinder end fill=gray!20,
			shape border rotate=90,
			aspect=0.25,
			draw
		}]
		\node [] (queue) {$\Pi_{b, d}$};
		\node [below of=queue] (join) {$\Join_{r.c = s.c}$};
		
		\node [below left of=join,xshift=-1cm] (l1) {$\sigma_{r.a = 'c'}$};
		\node [database, below of=l1] (l2) {\texttt{r}};
		
		\node [below right of=join,xshift=1cm] (r1) {$\sigma_{s.e = 2}$};
		\node [database,below of=r1] (r2) {\texttt{s}};
		
		\draw [<-] (queue) -- (join);
		\draw [<-] (join) -- (r1);
		\draw [<-] (r1) -- (r2);
		\draw [<-] (join) -- (l1);
		\draw [<-] (l1) -- (l2);
		\end{tikzpicture}
		\caption{Sample evaluation plan}\label{fig:subfig2}
	\end{subfigure}
\end{figure}

\section{Listings}
You can use listings to typeset source code. This template uses the \textit{listings} package. Wrap code inside the \texttt{lstlisting} environment and set the \textit{language} (e.g., Python, Java), \textit{caption}, and optional \textit{label} parameters. If the source code highlighting highlights the wrong keywords or misses keywords, use the \textit{deletekeywords} resp. \textit{morekeywords} parameters. Consult the package documentation for further information.

\begin{lstlisting}[float=htp, caption={Simple function implemented in Python}, label={lst:euclid}, language=Python, deletekeywords={}, morekeywords={}]
def add_one(n):
    return n+1
\end{lstlisting}

\section{Algorithms}
Some users might require specifying algorithms. This template uses the \textit{algorithm}, \textit{algorithmicx}, and \textit{algopseudocode} packages. Consult the respective manuals for further information. 
The example shows algorithm \ref{alg1}.

\begin{algorithm}[htb]
\footnotesize
\begin{algorithmic}[1]
    \State Step 1
    \State Step 2 
    \If {$(a != b)$}
  	    \State Step 3.1
    \Else 
  	    \State Step 3.2
    \EndIf
\end{algorithmic}
\caption{\label{alg1} Example algorithm}
\end{algorithm}

\section{Citations and Bibliography}
This template uses {BibTeX} for bibliographies. Of course, you have to maintain a clean \path{library.bib} file that caters all necessary attributes. References will appear in the order in which they appear in the text.

Citing in the text is done with the \textbackslash cite\{<citekey>\}.

\paragraph{Exemplary citations}

\begin{itemize}
	\item See \cite{Kist2021featurebased} for further information.
	\item ... like in \cite{Kist2021openhsv}.
\end{itemize}

