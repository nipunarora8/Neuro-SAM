\chapter{Conclusion}

This thesis introduced {Neuro-\gls{SAM}, a modular and interactive segmentation framework designed to address the challenges of dendrite and dendritic spine analysis in high-resolution 3D microscopy volumes. In contrast to earlier models that either relied on conventional \gls{CNN} architectures or generalized foundation models with minimal adaptation, Neuro-\gls{SAM} was purpose-built to handle the structural complexity, scale, and variability inherent to neural data. It blends several components to produce a robust and user friendly tool for segmentation of dendrites and dendritic spines. 

The path tracing module builds upon and improves the brightest-path-lib algorithm by incorporating waypoints, enabling interactive, reliable path reconstruction through densely packed dendritic fields. These paths then act as prompts for guiding the segmentation model, providing localized context through spatially sampled patches and precisely placed prompts. The dendrite segmentation module leverages a fine-tuned version of \gls{SAMv2} adapted with custom prompt strategies to generate accurate and continuous masks even in low-contrast or curved regions. Similarly, the spine segmentation module uses spatial prompts for segmentation, achieving high precision with minimal false positives.

Extensive quantitative evaluations show that Neuro-\gls{SAM} mostly outperforms \gls{SAM}, \gls{SAM}+\gls{LoRA}, and \gls{DeepD3} across both dendrite and spine segmentation tasks. It achieves high Dice and IoU scores across all expert raters and consensus masks and maintains good precision-recall tradeoffs. In qualitative comparisons, Neuro-\gls{SAM} produces cleaner, more interpretable masks with stronger alignment to human annotations, even in difficult conditions such as overlapping dendrites, dense spine clusters, and imaging noise. Furthermore, generalization experiments across multiple unseen datasets demonstrate the robustness of the pipeline, underscoring the effectiveness of its prompt-aware modular design. 

In summary, Neuro-\gls{SAM} demonstrates a robust, path-aware framework for detailed neural analysis, combining efficient path tracing, high-fidelity dendrite segmentation, and fine-grained spine segmentation. Through extensive quantitative and qualitative evaluations, in some cases it outperforms prior models across benchmark annotations and generalizes effectively to unseen datasets with diverse imaging characteristics. Its modular architecture, human-in-the-loop flexibility, and minimal reliance on dense annotations position it as a scalable solution for large-scale, interactive neuroanatomical mapping. These results not only validate the design of Neuro-\gls{SAM} but also establish its practical relevance for downstream neuroscience applications.
